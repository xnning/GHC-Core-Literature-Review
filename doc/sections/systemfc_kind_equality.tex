\section{Explicit Kind Equality}

System FC with Explicit Kind Equality, ICFP'13 \citep{weirich2013system}.

\subsection{Motivation}

System FC lacks kind equality proofs, as mentioned in Section~\ref{sec:fcpro}.
This paper presents an approach based on dependent type systems with
\hl{heterogeneous} equality and the \textit{Type-in-Type} axiom, yet it
preserves the metatheoretic properties of FC.

It enables

\begin{itemize}
\item Kind-indexed GADT: the datatype is indexed by both kind and type information.
  \begin{verbatim}
data TyRep :: \/ k. k -> * where
  TyInt    :: TyRep Int
  Ty Bool  :: TyRep Bool
  TyMaybe  :: TyRep Maybe
  TyApp    :: TyRep a -> TyRep b -> TyRep (a b)

zero :: \/ (a : *). TyRep a -> a
zero TyInt             = 0
zero TyBool            = False
zero (TyApp TyMaybe _) = Nothing
  \end{verbatim}
\item Promoted GADT: GADT are allowed to be used an as index.
  \begin{verbatim}
data Kind = Star | Arr Kind Kind

data Ty :: Kind -> * where
  TInt   :: Ty Star
  TBool  :: Ty Star
  TMaybe :: Ty (Arr Star Star)
  TApp   :: Ty (Arr k1 k2) -> Ty k1 -> Ty k2

data TyRep (k :: Kind) (t :: Ty k) where
  TyInt   :: TyRep Star TInt
  TyBool  :: TyRep Star TBool
  TyMaybe :: TyRep (Arr Star Star) TMaybe
  TyApp   :: TyRep (Arr k1 k2) a -> TyRep k1 b -> TyRep k2 (TApp a b)
  \end{verbatim}
\item Kind family
  \begin{verbatim}
kind family IK (k :: Kind)
kind instance IK Star = *
kind instance IK (Arr k1 k2) = IK k1 -> IK k2
  \end{verbatim}
\end{itemize}

\subsection{Notes}

\begin{itemize}
\item In this work, the syntax of types and kinds are unified, allowing us to
  reuse type coercions as kind coercions, with axoim $[[ star ]] : [[star]]$.\\
  $[[s]], [[k]] ::= [[\/a : k . s]] \mid ...$ \hl{$\mid [[\/ cx : prop . s]] \mid [[s |> co]] \mid
    [[s co]]$}\\
  The type of coercion $[[prop]]$ is now separated from types. Namely 
  coercion abstractions are separated from arrow types.\\
  $[[prop]] ::= [[s1 ~ s2]]$
\item Equalities are now \textit{heterogeneous}. In the definition of type
  equalities $[[s1 ~ s2]]$, the type $[[s1]]$ and $[[s2]]$ could have kinds
  $[[k1]]$ and $[[k2]]$ that have no syntactic relation to each other. A proof
  $[[co]]$ of $[[s1 ~ s2]]$ implies not only that $[[s1]]$ and $[[s2]]$ are
  equal, but also that their kinds are equal.
\item Coercions are irrelevant to both the operational semantics and type equivalence.
\item Important kinding rule:
  \[ \drule{fc-kind-KCast} \]
  Important coercion coherence rule:
  \[ \drule{fc-co-CT-COH} \]
  The use of kind coercions can be ignored when proving type equalities.
  \[ \drule{fc-co-CT-EXT} \]
  The new coercion form $[[kind co]]$ extracts the
  proof of $[[k1 ~ k2]]$ from $[[co]]$.
\item A tricky \verb|S_KPUSH| rule.
\end{itemize}

%%% Local Variables:
%%% mode: latex
%%% TeX-master: "../doc"
%%% org-ref-default-bibliography: "../doc.bib"
%%% End: