\section{$\FCpro$}
\label{sec:fcpro}

Giving Haskell a Promotion, TLDI'12 \citep{yorgey2012giving}.

\subsection{Motivation}

The \hl{kind system} in Haskell is too 1) permissive: type-level programming in
Haskell is almost entirely untyped, because the type system has too few kinds
($[[ star ]], [[ star -> star ]]$, and so on); 2) restrictive: It lacks
polymorphism.

This paper presents $\FCpro$, which extends \FC with
\begin{itemize}
\item Automatic promotion of datatypes to be kinds and data constructors to be
  types.
\begin{verbatim}
data Nat = Zero | Succ Nat
data Vec :: * -> Nat -> * where
  VNil  :: Vec a 'Zero
  VCons :: a -> Vec a n -> Vec a ('Succ n)
\end{verbatim}

Type \texttt{Nat} is used as a kind, and data constructors \texttt{Zero} and
\texttt{Succ} are used as types, with a quote notation to avoid ambiguity.

\item Kind polymorphism, for kinds, types, and terms.

\begin{verbatim}
data EqRefl a b where
  Refl :: EqRefl a a
\end{verbatim}

Previously, \verb|EqRefl|$:: [[star -> star -> star]]$, with kind polymorphism
we have \verb|EqRefl|$:: [[ \/ kx . kx -> kx -> star ]]$
  
\end{itemize}

\subsection{Notes}

\begin{itemize}
\item The formalization distinguish expressions, types, coercions and kinds, but
  in implementation they are \textit{combined}.
\item Expressions now include kind abstraction and kind application.
\item Kinds $[[k]] ::= [[star]] \mid [[k1 -> k2]] \mid$ \hl{$
    [[constraint]] \mid [[kx]] \mid [[\/kx. k]] \mid [[T kl]] $}, where $[[T]]$
  is promoted type constant.
\item Only one sort $[[GG |-k k : box ]]$
\item Types $[[s]] ::= ...$ \hl{$\mid [[K]] \mid [[\/ kx. s]] \mid [[s k]] \mid[[~]]$}, where
  $[[K]]$ is promoted data constructors, and $[[~]]$ is equality.
\item Important rules for promotion:
  \[ \drule{fc-kind-KLift} \]
  In this rule, a data constructor $[[K]]$ is treated as a type and has a
  kinding rule. $[[empty |- s ~> k]]$ turns a type into a kind.
  \[ \drule{kind-valid-KV-Lift} \]
  In this rule, a type constructor $[[T]]$ is treated as a kind constructor.
  This rule is relatively restrictive since the type of $[[T]]$ takes all
  arguments of kind $[[star]]$ and it needs to be fully saturated.
\item $\FCpro$ turns the equality from \FC into a type constructor with
  polymorphic kind:
  \[ \drule{fc-kind-KEq} \]
\item Coercions are homogeneous, having type $[[s1 ~ s2]]$, which has kind $[[constraint]]$.
\item Design principle: no kind equalities.
\end{itemize}




%%% Local Variables:
%%% mode: latex
%%% TeX-master: "../doc"
%%% org-ref-default-bibliography: "../doc.bib"
%%% End: